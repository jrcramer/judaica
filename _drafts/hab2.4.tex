\documentclass[a4paper,11pt]{article}
%\usepackage[landscape]{geometry}
\usepackage{wrapfig}
\usepackage[dutch]{babel}
\usepackage{xltxtra}
\usepackage{bidi}
\usepackage[strict,style=english]{csquotes}
\usepackage[style=authortitle-ibid,ibidtracker=true,isbn=false]{biblatex} %block=ragged
\bibliography{~/Dropbox/TeX/bib}

%\usepackage{bibleref}
%\input{~/Dropbox/TeX/bijbelboeken}
\newcommand{\ibibleverse}[1]{#1}

\defaultfontfeatures{Scale=MatchLowercase}
\newfontfamily{\heb}[Script=Hebrew]{EzraSIL} %Code2000,EzraSIL,SBLHebrew
\newfontfamily{\grk}[]{Gentium Plus}

\setlength{\parindent}{0pt}
\frenchspacing

%\setromanfont{Gentium} 

\newcommand{\bibref}[1]{{\footnotesize #1}}
\newcommand{\h}[1]{{\heb {\beginR{#1}\endR}}}
%\newcommand{\hft}[1]{{\h{\scriptsize #1}}}
\newcommand{\hp}[1]{\setRL {\heb{\beginR{#1}\endR}} \setLR}
		    
\newcommand{\q}[1]{\textit{\enquote{#1}}}

%\newcommand{\htab}[3]{
%\begin{table}[h]
%  \begin{tabular}{p{.6\textwidth} p{.4\textwidth}}
%	\hline {#1} & \hp{#2} \\ \hline
%  \end{tabular}
%  \raggedleft{#3}\\
%\end{table}}
	
\newcommand{\com}[2]{\par
\begin{wrapfigure}{r}{5.5cm}
  \vspace{-15pt}%
    \begin{tabular}{p{5.5cm}} #1 \end{tabular}%
  \vspace{-12pt}%
\end{wrapfigure}
\noindent #2\\}

%\renewcommand{\rmdefault}{\sfdefault}

\title{Gebruik van Hab 2:4 in de Mechilta}
\author{Jaap Cramer}
\date{\today}

\begin{document}

\maketitle

De Mechilta is een rabbijns commentaar op delen uit Exodus. In tractaat Beshallach 7, vinden we de bespreking van Exodus 14:26-31; het verhaal dat Israël door de Rode Zee trekt. Op het moment dat het volk veilig tussen de muren van zeewater is getrokken, valt het water terug, om de Egyptenaren die Israël achtervolgenden te verdrinken.
%\\
%NBV   26 De HEER zei tegen Mozes: ‘Strek je arm uit boven de zee; dan stroomt het water terug, over de Egyptenaren en over al hun wagens en ruiters.’ 27 Mozes gehoorzaamde, en toen de dageraad aanbrak, stroomde de zee terug naar haar gewone plaats. De Egyptenaren vluchtten het water tegemoet, de HEER dreef hen regelrecht de golven in. 28 Het terugstromende water overspoelde het hele leger van de farao, al zijn wagens en ruiters, die achter de Israëlieten aan de zee in gereden waren; niet een van hen bleef in leven. 29 Maar de Israëlieten waren dwars door de zee gegaan, over droog land, terwijl rechts en links van hen het water als een muur omhoogrees. 30-31 Zo redde de HEER de Israëlieten die dag uit de handen van de Egyptenaren. Toen ze de Egyptenaren dood langs de zee zagen liggen en het tot hen doordrong hoe krachtig de HEE	R tegen Egypte was opgetreden, kregen ze ontzag voor de HEER en stelden ze hun vertrouwen in hem en in zijn dienaar Mozes. 

Vers 31 eindigt met:\\
\com{ \hp{וַֽיַּאֲמִ֨ינוּ֙ בַּֽיהוָ֔ה וּבְמֹשֶׁ֖ה עַבְדּֽוֹ} }
{\q{En ze geloofden in de Heer en zijn dienaar Mozes.}}

Het commentaar bespreekt verschillende teksten die gaan over geloof.

% tijdje spreken ze over het verband dat er schijnbaar is tussen dat geloof, en zingen en de Heilige Geest mogen hebben.\\

Allereerst wordt het dubbele object in de zin besproken: (geloof in) de Heer en in zijn dienaar Mozes. De vraag klinkt of geloof in Mozes ook niet geloof in  God impliceert. Was het niet genoeg om te zeggen dat het volk Mozes vertrouwde? \\ %Mechilta, vol 1 p252
Op de achtergrond speelt de overtuiging dat er in de bijbel geen woord te veel staat. Alles wat schijnbaar dubbel, of zelfs overbodig lijkt, moet iets te zeggen hebben.\\
Het antwoord is dat de implicatie expliciet wordt genoemd om te leren dat vertrouwen op de herder van Israel hetzelfde is als vetrouwen op/geloven in de schepper van hemel en aarde.\footnote{Het hebreeuwse werkwoord \h{ןמא} betekent \textit{geloven in, vertrouwen op}. In het Nederlands zit er een verschil in connotatie.}\\
In Numeri 21:5 is een soortgelijk geval: \q{Het volk sprak tegen God en tegen Mozes}. Weer dezelfde vraag, maar dan andersom: impliceert spreken tegen God niet dat dat via Mozes ging? En nu is het antwoord. \q{Dit leert dat de herder van Israel tegenspreken hetzelfde is als de schepper van het al tegen te spreken. }

Geloven in of vertrouwen op God en op Mozes horen bij elkaar.
Het commentaar gaat verder op het thema van geloven. \q{For as a reward for the faith with which Israel believed in God, the Holy Spirit rested upon them and they uttered a song, \ldots} \\ %שבשכר אמנה שהאמינו ישראל ביי שרתה עליהם רוח הקדש ואמרו שירה
Hier worden twee nieuwe thema's aangesneden, die ook met elkaar hebben te maken, namelijk de Heilige Geest en zingen.\\
Rabbi Nehemiah vraagt eens \q{Hoe kun je bewijzen dat het houden van slechts een van de geboden met waar geloof voldoende is om het waard te zijn dat de Heilige Geest op je rust?} en dan verwijst hij naar de tekst die het commentaar bespreekt: Exodus 14:3 en 15:1. \\ 
Hij noemt nog meer teksten, maar ook het geloof van Abraham uit Gen. 15:6 is een voorbeeld van het houden van een elkel gebod (namelijk geloven) dat het hem waard maakte de Heilige Geest te hebben.


% \ibibleverse{Psalms}(118:20) (Jesaja 26:2)\\
% \ibibleverse{Psalms}(92:2,3,5)\\
% %\ibibleverse{2 Kronieken}(20:20)\\
% \ibibleverse{Jeremia}(5:3)\\
% \ibibleverse{Habbakuk}(2:4)\\
% \ibibleverse{Klaagliederen}(3:23) \h{חֲדָשִׁים לַבְּקָרִים רַבָּה אֱמוּנָתֶךָ׃} \\
% \ibibleverse{Hooglied}(4:8)\\
% \ibibleverse{Hosea}(2:21,22)

Opvallend in deze rij citaten is Habbakuk 2:4, een tekst die door Paulus wordt geciteerd in Rom. 1:17 en Gal. 3:11. Paulus gebruikt deze tekst in Galaten 3 om het contrast tussen wet en geloof te demonstreren. Hij citeert eerst Deut. 27:26 (vgl Jer 11:3): \q{Vervloekt wie de voorschriften van deze Wet niet hooghoudt en ze niet volbrengt.} Dat dit hooghouden ook op het daadwerkelijk doen neerkomt, en dat bovendien je leven daarvan afhangt, laat Paulus zien door in Galaten 3:12 Leviticus 18:5 te citeren.\\
Paulus stelt dit doen tegenover geloof. Je wordt rechtvaardig door geloof. Dat is de functie van Habbakuk 2:4 in Paulus' betoog.

Binnen protestantse theologie is het Sola Fide een belangrijke kern. Tegen 

\section{Rabbi Pappias}
Eerder in dit tractaat

Pappias geeft een exegese die dan door R. Akiva wordt veroordeelt. Het gaat om de teksten: 
Hoogl 1:9 in verbinding met Hab 3:15; Job 23:13; Gen 3:22; Ps 106:20



\addcontentsline{toc}{section}{Literatuur}
\defbibheading{lit}{\section*{Literatuur}}
\printbibliography[heading=lit]

\end{document}
