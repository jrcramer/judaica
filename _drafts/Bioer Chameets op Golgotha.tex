\documentclass[a4paper,11pt]{article}
%\usepackage[landscape]{geometry}
\usepackage{wrapfig}
\usepackage[dutch]{babel}
\usepackage{xltxtra}
\usepackage{bidi}
\usepackage[strict,style=english]{csquotes}
\usepackage[style=authortitle-ibid,ibidtracker=true,isbn=false]{biblatex} %block=ragged
\bibliography{~/Dropbox/TeX/bib}

\newcommand{\ibibleverse}[1]{#1}

\defaultfontfeatures{Scale=MatchLowercase}
\newfontfamily{\heb}[Script=Hebrew]{EzraSIL} %Code2000,EzraSIL,SBLHebrew
\newfontfamily{\grk}[]{Gentium Plus}

\setlength{\parindent}{0pt}
\frenchspacing

%\setromanfont{Gentium} 

\newcommand{\bibref}[1]{{\footnotesize #1}}
\newcommand{\h}[1]{{\heb {\beginR{#1}\endR}}}
%\newcommand{\hft}[1]{{\h{\scriptsize #1}}}
\newcommand{\hp}[1]{\setRL {\heb{\beginR{#1}\endR}} \setLR}
		    
\newcommand{\q}[1]{\textit{\enquote{#1}}}

	
\newcommand{\com}[2]{\par
\begin{wrapfigure}{r}{5.5cm}
  \vspace{-15pt}%
    \begin{tabular}{p{5.5cm}} #1 \end{tabular}%
  \vspace{-12pt}%
\end{wrapfigure}
\noindent #2\\}

\title{Bioer Chameets op Golgotha}
\author{Jaap Cramer}
\date{\today}

\begin{document}

\maketitle

Het Joodse feest van Pesach, de uittocht uit Egypte, wordt in het Nieuwe Testament gebruikt als model om te begrijpen wat Jezus aan het kruis heeft gedaan. Pesach bevat diverse elementen, die een christelijke duiding krijgen. Er is bevrijding uit Egypte, beeld van onze bevrijding van de zonde. Er is een paaslam dat geslacht moet worden; Jezus is dat paaslam. Er worden ongezuurde broden gegeten en wijn gedonken, en bij de instelling zegt Jezus dat het brood en die wijn, zijn lichaam en bloed is.

Jezus is zowel het lam, als het brood, als de wijn. De beelden lopen door elkaar.
Dit korte onderzoek wil vooral aandacht geven aan het ongezuurde brood, welke joodse achtergronden mogelijk hebben meegeklonken, en hoe dat in het Nieuwe Testament meeklinkt.



\subsection*{Grote schoonmaak}
Voordat een Joodse familie Pesach kan vieren moet eerst het huis worden schoongemaakt van alle zuurdesem, Chameets.
In Exodus 12:15 staat het gebod om op de eerste dag het desem het huis uit moet zijn.
%Mechilta, Pischa H8 (band 1 p 60), verwijst bovendien naar Deut 16:3, waar wordt uigelegd, dat niet alles matze moet zijn (ongezuurd), maar dat het gaat om het ongezuurd brood. De Mechilta werkt verder uit dat dit alleen geldt voor graansoorten die kunnen verzuren/vergisten. De 5 waar dat voor geldt, en waar dus Matze van gemaakt mag worden is 'wheat, barley, spelt, oats and rye'. Dit geldt net voor 'rice, millet, poppyseed, sesame, and legumes' omdat deze niet kunnen verzuren maar alleen verrotten kan het gebod op ongezuurd-zijn niet daarop slaan.  
In de Joodse traditie wordt uitgelegd dat dit vers zou kunnen betekenen dat de grote schoonmaak op de eerste dag van het festival moet zijn. Maar dit kan niet bedoeld zijn, betoogt de Mechilta (Pischa 8), omdat in Ex 34:25 staat dat het lam niet geslacht mag worden met en desem in de buurt, en dit gebeurt op de eerste dag, dus moet het huis voor die tijd schoon zijn. Bovendien is de manier van verwijderen verbranden, wat niet op het festival zelf gedaan mag worden. 

\subsection*{Zuurdesem in het Nieuwe Testament}
In het nieuwe testament komt het desem meerdere malen voor.\\
Jezus gebruikt het als positief beeld van het Koninkrijk. Zoals het klein begint, trekt het het hele deeg door. (Mat 13:33, Luk 13:21)

Negatief gebruikt Jezus het als beeld voor wat er niet deugd aan de Farizeeen en Sadduceeen/Herodianen. Jezus waarschuwd voor hun zuurdesem. (Mat 16,6vv, Mk 8:15, Luk 12:1)

Paulus gebruikt het beeld tweemaal. In Galaten 5:9 gaat het over de besmettelijke functie. Er is maar weinig nodig. In 1 Kor 5:6 wordt dit beeld nog wat verder uitgewerkt. Ook hier staat het in de context van zonde. Maar daarbij geeft Paulus een gebod om het op te ruimen. 
\q{Doe de oude desem weg en wees als nieuw deeg. U bent immers als ongedesemd brood omdat ons pesachlam, Christus, is geslacht. Laten we daarom het feest niet vieren met de oude desem van kwaad en ontucht, maar met het ongedesemde brood van reinheid en waarheid.}

Paulus gebruikt het beeld van het opruimen van het desem, rondom het Pesach feest. Het desem staat symbool voor de zonde.

Johannes noemt expliciet dat het voorbereidingsdag is. Dat brengt enkele commentaren ertoe dat te verbinden met de erev pesach, met alle schoonmaakrituelen hierboven beschreven.\\
Hoewel de chronologie tamelijk ingewikkeld is, en het meer voor de hand ligt om bij het woord voorbereidingsdag aan de dag voor Sabbat (d.i. de goede vrijdag) te denken, en niet aan de dag voorafgaand aan de pesachviering, is het beeld van het schoonmaken zo fascinerend en rechtvaardigt Paulus' latere gebruik om te zoeken naar tekstuele aanwijzingen in de evangelieeën.

\subsection*{Drinken}


Mat 27:48
καὶ εὐθέως δραμὼν εἷς  ⸋ἐξ αὐτῶν⸌ καὶ λαβὼν σπόγγον πλήσας τε ὄξους καὶ περιθεὶς καλάμῳ ἐπότιζεν αὐτόν.

Mk 15:36
δραμὼν δέ τις  ⸂[καὶ] γεμίσας⸃ σπόγγον ὄξους περιθεὶς καλάμῳ  ⸋ἐπότιζεν αὐτόν

Luk 23:36
ἐνέπαιξαν δὲ αὐτῷ καὶ οἱ στρατιῶται προσερχόμενοι,  ⸂ὄξος προσφέροντες αὐτῷ⸃

Joh 19:29
σκεῦος  ⸆ ἔκειτο ὄξους μεστόν·  ⸂σπόγγον οὖν μεστὸν τοῦ ὄξους ὑσσώπῳ περιθέντες⸃ προσήνεγκαν αὐτοῦ τῷ στόματι.


OXOS in alle vier genoemd

Deze zure wijn die Jezus drinkt is niet dezelfde als die Matteus en Marcus eerder noemen.

(Mat 27:34 (οἶνον μετὰ χολῆς μεμιγμένον) en Markus 15:23 (ἐσμυρνισμένον οἶνον) noemen daarnaast ook nog een gemengde wijn, maar daar wil Jezus niet van. (vBruggen, Markus, 375))



Deze optie geeft misschien wat achtergrond bij de woorden van Johannes, dat Jezus de schrift wil vervullen als hij zijn dortst uitroept. De Verwijzingen naar ps 68 (vHouwelingen, Johannes, 374), en ook de beschrijving van droogte in Ps 22

Jezus wordt tot zonde gemaakt, en neemt de gegiste drank tot zich; zoals de Chameets beeld van zonde was, zo ook de Chomets.
Jezus wordt het zuurdesem om helemaal verbrand te worden, 






%ber 17a
% and R. Alexander used to add at the conclusion of his prayer : 
% 6
% Lord of the universe ! It is revealed and known before Thee that it is our will to perform Thy will ; but what stands in the way?
% 7
% The leaven that is in the dough[3] and the servitude of the kingdoms[4]. 
% 8
% May it be Thy will to deliver us from their hand[5], so that we may again perform the statutes of Thy will with a perfect heart. 


\end{document}